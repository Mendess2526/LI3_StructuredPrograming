\documentclass[11pt,a4paper]{report}
\usepackage[utf8]{inputenc}
\usepackage[portuguese]{babel}
\usepackage{titlesec}
\usepackage{graphicx}
\usepackage{float}
\usepackage{array}
\usepackage{multirow}
\usepackage{geometry}
\usepackage{minted}
\usepackage{pdflscape}
\usepackage[titletoc]{appendix}
\geometry{
 a4paper,
 top=3cm,
 bottom=2.5cm,
 left=3cm,
 right=2.5cm
}
\titleformat{\chapter}{\normalfont\huge}{\thechapter.}{10pt}{\huge}
\begin{document}
\begin{titlepage}
    \center
    \vspace*{4.0cm}
    {\huge {\bf Universidade do Minho}}\\[0.4cm]
    \vspace{3.0cm}
    \textsc{\huge{Processamento de xml}}\\[0.5cm] %TODO melhorar isto
    \vspace{3.0cm}
    \textsc{\huge{Mestrado Integrado em Engenharia Informática}}\\[0.5cm]
    \vspace{2.0cm}
    \textsc{Laboratorios de Informática 3}\\[0.5cm]
    \textsc{(2º Ano, 2º Semestre, 2017/2018)}\\[0.5cm]
    \vspace{1.5cm}
    \begin{flushleft}
        A79003 \,\,\,Pedro Mendes Félix da Costa
        \vspace{0.2cm}

        A80453 \,\,\,Barbara Andreia Cardoso Ferreira
        \vspace{0.2cm}

        AXXXXX \,\,\,Ana Catarina %TODO
    \end{flushleft}
        \vspace{1cm}
    \begin{flushright}
        Braga

        Abril 2018
    \end{flushright}

\end{titlepage}

\chapter{Resumo}


\tableofcontents %indice

\chapter{Introdução}


\chapter{Descrição do Problema}
    Falar do Stackoverflow

    Analizar queries


    Juntando as informações chegamos a conclusão que..
    \section{Users}
    Para represantar um user guardamos os seguintes atributos:
    \begin{itemize}
            \item Id
            \item bio
            \item
            \item
    \end{itemize}

    \section{Posts}
    Inicialmente foi pensada a utilização de uma estrutura generica.

        \subsection{Questions}

        \subsection{Answers}

    \section{Tags}


\chapter{Solução}

    \section{}

        \subsection{Questions}
        Metemos uma lista de respostas para reponder a query 9
        Guardamos as tags como uma unica string em que cada tag esta entre
        $< >$


        \subsection{Answers}
        Metemos um apontador para a questao a que responde

        \subsection{Users}
        Metemos o numero de posts para responder a query 2

        Metemos uma lista de posts do utilizador para responder a query 9
        Foi preciso definir uma estura post que guarda uma questão ou resposta para
        que pudessem ambas guardadas na mesma lista.

        \subsection{Tags}

    \section{Estruturas de Dados}

        \subsection{Hash Tables}

        A hash table de tags usa a string da tag como chave, para obter o id.
        Usado para responder à query 11.

        \subsection{Calendario}
        Foi pensada a utilização de arvores binarias ordenadas por data, mas isto
        revelou-se que isto era ineficiente.

        Foi preciso definir um DateTime
        Explicar calendario.

\chapter{Conclusões e Trabalho Futuro}


\end{document}
