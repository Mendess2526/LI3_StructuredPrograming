\documentclass[10pt,a4paper]{article}
\usepackage[utf8]{inputenc}
\usepackage[portuguese]{babel}
\usepackage{titlesec}
\usepackage{graphicx}
\usepackage{indentfirst}
\usepackage{enumerate}
\usepackage{float}
\usepackage{array}
\usepackage{tikz}
\usepackage{multirow}
\usepackage{multicol}
\usepackage{geometry}
\usepackage[cache=false]{minted}
\usepackage{pdflscape}
\usepackage[titletoc]{appendix}
\usepackage{hyperref}
\geometry{
 a4paper,
 top=1cm,
 bottom=1cm,
 left=3cm,
 right=3cm
}
\addto\captionsportuguese{
      \renewcommand{\contentsname}
          {Índice}
}

\begin{document}
\begin{titlepage}
    \center
    {\huge {\bf Universidade do Minho}}\\[0.4cm]
    \vspace{3.0cm}
    \textsc{\huge{Processamento de xml}}\\[0.5cm] %TODO melhorar isto
    \vspace{3.0cm}
    \textsc{\huge{Mestrado Integrado em Engenharia Informática}}\\[0.5cm]
    \vspace{2.0cm}
    \textsc{Laboratórios de Informática 3}\\[0.5cm]
    \textsc{(2º Ano, 2º Semestre, 2017/2018)}\\[0.5cm]
    \vspace{1.5cm}
    \begin{flushleft}
        A79003 \,\,\,Pedro Mendes Félix da Costa
        \vspace{0.2cm}

        A80453 \,\,\,Bárbara Andreia Cardoso Ferreira
    \end{flushleft}
        \vspace{1cm}
    \begin{flushright}
        Braga

        Maio 2018
    \end{flushright}

\end{titlepage}

\tableofcontents
\clearpage

\section{Introdução}
    Este trabalho tem como objetivo o processamento de qualquer \textit{dump}
    da base de dados do site \href{www.stackoverflow.com}{StackOverflow} para
    responder a queries de forma eficiente, aplicando conhecimentos de
    algoritemia e programação imperativa.

\section{Descrição do Problema}
    Fazendo uma análise às queries, decidimos que seria necessário representar
    as seguintes entidades:

    \subsection{Posts}
    Para representar os \textbf{posts} dividimos os atributos por duas
     sub estruturas, cujos atributos de cada uma das sub estruturas são:
        \subsubsection{Questões}
        \begin{multicols}{2}
        \begin{itemize}
            \item Id
            \item Score
            \item Data
            \item Título da questão
            \item Id do autor da questão
            \item Nome do autor da questão
            \item Número de respostas
            \item Lista das respostas
            \item Tags
        \end{itemize}
        \end{multicols}

        \subsubsection{Respostas}
        \begin{multicols}{2}
        \begin{itemize}
            \item Id
            \item Score
            \item Data
            \item Número de comentários
            \item Id do autor da resposta
            \item Id da questão a que responde
            \item Nome do autor da resposta
            \item Referência da questão a que responde
        \end{itemize}
        \end{multicols}
    Além dos dados fornecidos diretamente pelos ficheiros xml, decidimos também
    guardar na \textbf{questão} a lista das \textbf{respostas} de cada questão,
    bem como uma contagem destas. Na \textbf{resposta}, inversamente, guardamos
    uma referência para a pergunta a que esta responde. Com estas informações
    extra as pesquisas que envolvem relacionar estas duas entidades tornam-se
    mais eficientes.

        \subsubsection{Post}
        Ambas estas estruturas, questões e respostas, podem ser representadas de
        forma abstrata por um \textbf{Post}.

        Esta estrutura é necessária, pois o utilizador precisa de guardar
        todas as questões e respostas, sem ser necessário uma distinção entre
        os dois tipos.

    \subsection{Utilizadores}
    Para representar os \textbf{utilizadores} guardamos os seguintes atributos:
    \begin{multicols}{3}
    \begin{itemize}
            \item Id
            \item Biografia
            \item Nome
            \item Reputação
            \item Número de posts
            \item Lista dos posts
    \end{itemize}
    \end{multicols}
    Mais uma vez foram guardadas mais informações para além das disponibilizadas
    diretamente pelo xml. Foi guardado o número de \textbf{posts} do utilizador
    (para determinar os utilizadores mais ativos de forma mais rápida) e a lista
    destes para permitir pesquisas mais rápidas.

\section{Estruturas de Dados}
    \subsection{Tipo concreto de dados}
    Para armazenar as entidades descritas acima foi implementado um TCD que
    as armazena de diferentes formas.
    \begin{minted}{C}
struct TCD_community{
    QUESTIONS_HTABLE questions;
    ANSWERS_HTABLE answers;
    SO_USERS_HTABLE users;
    TAGS_HTABLE tags;
    CALENDARIO calendarioQuestions;
    CALENDARIO calendarioAnswers;
};
    \end{minted}
    \subsection{Hashtables}
        Todas as entidades são armazenadas numa tabela de hash pois para todas
        são necessárias pesquisas por id (ou nome no caso das tags).

        \subsubsection{Tags}
        A tabela de hash das \textit{tags} serve para criar uma associação
        $Nome \to Id$ visto que as questões guardam uma lista com os nomes das
        tags e para responder à query 11 é necessário obter os ids das mesmas.

    \subsection{Calendário}
        Foi pensada a utilização de árvores binárias ordenadas por data,
        mas isto foi considerado ineficiente quando comparada à solução
        escolhida. Foi então concebida uma estrutura à qual foi dado o nome
        de \textbf{Calendário}.

        Esta estrutura permite:
        \begin{itemize}
                \item Guardar qualquer objeto desde que seja
                      passada uma data associada ao mesmo.
                \item Iterar sobre os objetos, dado um intervalo de tempo,
                      por ordem cronológica normal ou inversa, conforme a
                      ordem dos argumentos.
        \end{itemize}

        A estrutura em si consiste numa árvore n-ària (rose tree) especializada.

        No primeiro nível temos uma lista de \textbf{anos}. Cada um destes,
        é constituído por uma lista de \textbf{meses} que são constituídos por
        uma lista de \textbf{dias}, cujo tamanho varia entre 29 e 31. Cada
        \textbf{dia} é constituido por 24 \textbf{horas}, e cada uma destas horas
        é uma lista ligada de elementos ordenada por data.

        \begin{tikzpicture}[>=latex]
\matrix[mymat,anchor=west,row 2/.style={nodes=draw}]
at (0,0)
(mat1)
{
  0       &     1       &     (...)       &     7 \\
  * & * & (...) & ** \\
};
\matrix[mymat,anchor=west, row 2/.style={nodes=draw}]
at (0,-3)
(mat3)
{
  0 & 1 & 2 & 3 & 4 & 5 & (...) & 371 \\
  * & * & * & * & * & * & (...) & * \\
};

  \node[above=0pt of mat1]
  (cella) {Array de anos};

  \node[above=-0pt of mat3]
  (cella) {Array de meses e dias};

  \node [matrix,draw] at (0,-5) (mat4){Post \\}
child {node[matrix,draw] {Post \\}}
child {node[matrix,draw] {Post \\}};

  \node[above=0pt of mat4]
  (cella) {GSequence de Posts};
\begin{scope}[shorten <= -2pt]
\draw[*->]
  (mat1-2-1.south) -- (mat3-1-1.north);
\draw[*->]
  (mat3-2-1.south) -- (mat4.north);
\end{scope}
\end{tikzpicture}


        A utilização de uma lista ligada para representar uma hora é mais
        eficiente, pois é necessário fazer inserções ordenadas, que são, em quase
        todos os casos, feitas à cabeça devido aos posts serem inseridos quase
        cronologicamente.

        \subsubsection{DateTime}
        Para a implementação da estrutura \textbf{Calendário} foi necessário
        extender o \textbf{Date} para incluir a hora, minuto, segundo e
        milissegundo da data do post. Foi então criado o \textbf{DateTime}.
    \begin{minted}{C}
struct _dateTime{
    int year, month, day;
    int hours, minutes, seconds, milisseconds;
};
    \end{minted}

    \subsection{String Rose Tree}
    Para auxiliar à resolução da query 11 foi implementada uma estrutura para
    contar \textit{strings}. Esta, consiste numa árvore n-ária em que qualquer
    string representa um caminho único sobre a árvore. Assim, quando é inserida
    uma string, efetivamente, é guardado o número de vezes que esse caminho é
    percorrido.

\section{Modularização Funcional e Resolução das queries}
    Para aceder aos dados da estrutura principal foi definida uma API
    simples que permite:
    \begin{enumerate}[1.]
        \item Pesquisas por id de \textbf{questões}, \textbf{respostas} e
        \textbf{utilizadores}.
        \item Pesquisas de ids de \textbf{tags} dada a designação.
        \item Pesquisa de listas, ordenadas por qualquer critério, de
        \textbf{utilizadores}, \textbf{questões} e \textbf{respostas}.
        \item Pesquisas de \textbf{questões} filtradas por qualquer critério.
        \item Pesquisas genéricas de \textbf{questões}/\textbf{respostas}
        num determinado intervalo de tempo.
    \end{enumerate}

    Com estas funções a resolução da maioria das queries mostrou-se
    trivial.

    Para as queries que necessitam de pesquisas por ID (queries: 1, 5, 9 10)
    são conseguidas por \textbf{1}.

    Para as queries que necessitam de pesquisas de utilizadores ordenados
    (queries: 2, 11) são conseguidas atravez de \textbf{3} bastando
    fornecer uma função de comparação para definir o critério de ordenação.

    Para queries que necessitam de pesquisas de
    \textbf{perguntas}/\textbf{respostas} num intervalo de tempo ordenadas
    (queries: 6 e 7) são também conseguidas atravez de \textbf{3}. No caso de não
    ser necessária a filtragem do intervalo de tempo simplesmente passamos as
    datas maximas definidas pelas macros \mintinline{C}{dateTime_get_epoch()} e
    \mintinline{C}{dateTime_get_year2038()} em $dateTime.h$.

    Para queries que necessitam de listas de questões que obedeçam a um
    determinado critério (queries: 4, 8), são conseguidas atravez de \textbf{4}.

    Nos caso em que estes metodos especializados não necessários (query: 3)
    fizemos uso de uma iteração genérica (\textbf{5}).

    Para a resolução da query 11 foi implementada uma estrutura para contar tags,
    para que esta contagem fosse eficiente. Com isto o unico trabalho necessário
    foi obter as \textbf{tags} das \textbf{quetões} dos \textbf{utilizadores}
    com melhor reputação (\textbf{1}). Depois abtemos a lista ordenada por
    ocorrencia das tags e atravez de \textbf{2} obtemos os ids das mesmas.

\section{Conclusões e Trabalho Futuro}
    Em conclusão, o grupo considera que o trabalho foi realizado na sua
    totalidade de forma eficiente e correta, respondendo a todas as queries.

    Um aspecto que poderia ser melhorado é a ordenação de utilizadores. Estes
    foram guardados apenas numa tabela de hash e quando é necessária um lista
    ordenada dos mesmos, esta tem de ser percorrida na sua totalidade. Esta
    decisão, centrou-se no facto de que nenhuma única ordenação se apresenta
    particularmente vantajosa, face às demais.

\end{document}
