\documentclass[10pt,a4paper]{article}
\usepackage[utf8]{inputenc}
\usepackage[portuguese]{babel}
\usepackage{titlesec}
\usepackage{graphicx}
\usepackage{indentfirst}
\usepackage{float}
\usepackage{array}
\usepackage{tikz}
\usepackage{multirow}
\usepackage{multicol}
\usepackage{geometry}
\usepackage[cache=false]{minted}
\usepackage{pdflscape}
\usepackage[titletoc]{appendix}
\usepackage{hyperref}
\geometry{
 a4paper,
 top=1cm,
 bottom=1cm,
 left=3cm,
 right=3cm
}
\addto\captionsportuguese{
      \renewcommand{\contentsname}
          {Índice}
}
\begin{document}
\begin{titlepage}
    \center
    {\huge {\bf Universidade do Minho}}\\[0.4cm]
    \vspace{3.0cm}
    \textsc{\huge{Processamento de xml}}\\[0.5cm] %TODO melhorar isto
    \vspace{3.0cm}
    \textsc{\huge{Mestrado Integrado em Engenharia Informática}}\\[0.5cm]
    \vspace{2.0cm}
    \textsc{Laboratorios de Informática 3}\\[0.5cm]
    \textsc{(2º Ano, 2º Semestre, 2017/2018)}\\[0.5cm]
    \vspace{1.5cm}
    \begin{flushleft}
        A79003 \,\,\,Pedro Mendes Félix da Costa
        \vspace{0.2cm}

        A80453 \,\,\,Bárbara Andreia Cardoso Ferreira
    \end{flushleft}
        \vspace{1cm}
    \begin{flushright}
        Braga

        Abril 2018
    \end{flushright}

\end{titlepage}

\tableofcontents
\clearpage

\section{Introdução}
    Este trabalho tem como objetivo o processamento de qualquer \textit{dump}
    da base de dados do site \href{www.stackoverflow.com}{StackOverflow} para
    responder a queries de forma eficiente, aplicando conhecimentos de
    algoritemia e programação imperativa.

\section{Descrição do Problema}
    Analisando as queries decidimos que seria necessário representar
    as seguintes entidades:

    \subsection{Posts}
    Para representar os \textbf{posts} dividimos os atributos por duas
     sub estruturas, cujos atributos são:
        \subsubsection{Questões}
        \begin{multicols}{2}
        \begin{itemize}
            \item Id
            \item Score
            \item Data
            \item Título da questão
            \item Id do autor da questão
            \item Nome do autor da questão
            \item Número de respostas
            \item Lista das respostas
            \item Tags
        \end{itemize}
        \end{multicols}

        \subsubsection{Respostas}
        \begin{multicols}{2}
        \begin{itemize}
            \item Id
            \item Score
            \item Data
            \item Número de comentários
            \item Id do autor da resposta
            \item Id da questão a que responde
            \item Nome do autor da resposta
            \item Referência da questão a que responde
        \end{itemize}
        \end{multicols}
    Além dos dados fornecidos diretamente pelos ficheiros xml, foi tb guardado na
    \textbf{questão} a lista da \textbf{respostas} desta bem como uma contagem
    destas, a \textbf{resposta}, inversamente, guarda uma referência para a
    pergunta a que responde. Com estas informações extra as pesquisas que
    envolvem relacionar estas duas entidades tornam-se muito eficientes.

        \subsubsection{Posts}
        Ambas estas estruturas podem ser represantadas de forma comun por um
        \textbf{Post}. Esta estrutura é necessária para quando é preciso guardar
        ambas independentemente do tipo que tem, como é o caso do utilizador.

    \subsection{Utilizadores}
    Para representar os \textbf{utilizadores} guardamos os seguintes atributos:
    \begin{multicols}{3}
    \begin{itemize}
            \item Id
            \item Biografia
            \item Nome
            \item Reputação
            \item Número de posts
            \item Lista dos posts
    \end{itemize}
    \end{multicols}
    Mais uma vez foram guardadas mais informações para além das disponibilizadas
    diretamente pelo xml. Foi guardado o número de \textbf{posts} do utilizador
    (para determinar os utilizadores mais ativos) e a lista destes para permitir
    pesquisas mais rápidas.

\section{Estruturas de Dados}
    \subsection{Tipo concreto de dados}
    Para armazenar as entidades descritas acima foi implementado um TCD que
    as armazena de diferentes formas.
    \begin{minted}{C}
struct TCD_community{
    QUESTIONS_HTABLE questions;
    ANSWERS_HTABLE answers;
    SO_USERS_HTABLE users;
    TAGS_HTABLE tags;
    CALENDARIO calendarioQuestions;
    CALENDARIO calendarioAnswers;
};
    \end{minted}
    \subsection{Hashtables}
        Todas as entidades são armazenadas numa tabela de hash pois para todas
        são necessárias pesquisas por id (ou nome no caso das tags).

        \subsubsection{Tags}
        A tabela de hash das \textit{tags} serve para criar uma associação
        $Nome \to Id$ visto que as questões guardam uma lista com os nomes das
        tags e para responder à query 11 é necessário obter os ids das mesmas.

    \subsection{Calendário}
        Foi pensada a utilização de árvores binárias ordenadas por data,
        mas isto foi considerado ineficiente quando comparada à solução
        escolhida. Foi então concebida uma estrutura a qual demos o nome
        de \textbf{Calendário}.

        Esta estrutura permite:
        \begin{itemize}
                \item Guardar qualquer objeto desde que seja
                      passada uma data associada ao mesmo.
                \item Iterar sobre os objetos dado um intervalo de tempo
                      por ordem cronológica normal ou inversa, conforme a
                      ordem dos argumentos.
        \end{itemize}
        \subsubsection{DateTime}
        Para a implementação deste foi necessário extender o \textbf{Date} para
        incluir a hora, minuto, segundo e milissegundo do post. Foi então foi
        criado o \textbf{DateTime}
    \begin{minted}{C}
struct _dateTime{
    int year, month, day;
    int hours, minutes, seconds, milisseconds;
};
    \end{minted}
        A estrutura em si consiste numa àrvore n-ària (rose tree) especializada.
        No primeiro nivel temos uma lista de \textbf{anos}. Cada um destes
        \textbf{anos} é por sí uma lista de \textbf{meses} constituidos por
        uma lista de \textbf{dias} (de tamanho entre 29 e 31) e cada \textbf{dia}
        é constituido por 24 \textbf{horas}, cada uma destas uma lista ligada de
        elementos ordenada por data.
    %\begin{tikzpicture}[>=latex]
\matrix[mymat,anchor=west,row 2/.style={nodes=draw}]
at (0,0)
(mat1)
{
  0       &     1       &     (...)       &     7 \\
  * & * & (...) & ** \\
};
\matrix[mymat,anchor=west, row 2/.style={nodes=draw}]
at (0,-3)
(mat3)
{
  0 & 1 & 2 & 3 & 4 & 5 & (...) & 371 \\
  * & * & * & * & * & * & (...) & * \\
};

  \node[above=0pt of mat1]
  (cella) {Array de anos};

  \node[above=-0pt of mat3]
  (cella) {Array de meses e dias};

  \node [matrix,draw] at (0,-5) (mat4){Post \\}
child {node[matrix,draw] {Post \\}}
child {node[matrix,draw] {Post \\}};

  \node[above=0pt of mat4]
  (cella) {GSequence de Posts};
\begin{scope}[shorten <= -2pt]
\draw[*->]
  (mat1-2-1.south) -- (mat3-1-1.north);
\draw[*->]
  (mat3-2-1.south) -- (mat4.north);
\end{scope}
\end{tikzpicture}

        A escolha de utilização de uma lista ligada para a representar uma hora
        foi devido às multiplas inserções ordenadas feitas sobre esta que, na
        maior parte dos casos é feita à cabeça devido à natureza quase ordenada
        dos posts nos ficheiros.

    \subsection{String Rose Tree}
    Para auxiliar à resolução da query 11 foi implementada uma estrutura para
    contar \textit{strings}. Ésta consiste numa àrvore n-ária em que qualquer
    string representa um caminho único sobre a àrvore. Assim, quando é inserida
    uma string, efetivamente é guardado o número de vezes que esse caminho é
    precorrido.

\section{Modularização Funcional e Resolução das queries}
    Para aceder aos dados da estrutura principal foi definida uma API
    simples que permite:
    \begin{itemize}
        \item Pesquisas por id de questões, respostas, utilizadores.
        \item Pesquisas de ids de tags dada a designação.
        \item Pesquisa de listas, ordenadas por qualquer critério, de
              utilizadores, questões e respostas.
        \item Pesquisa de questões filtradas por qualquer critério.
    \end{itemize}

    Com estas funções a resolução da maioria das queries mostrou-se
    trivial.

    Para a resolução da query 11 foi implementada uma estrutura para contar tags,
    para que esta contagem destas fosse eficiente.

\section{Conclusões e Trabalho Futuro}
    Em conclusão, o grupo considera que o trabalho foi realizado na sua
    totalidade de forma eficiente e correta, respondendo a todas as queries.

    Um aspecto que poderia ser melhorado é a ordenação de utilizadores. Estes
    foram guardados apenas numa tabela de hash e quando é necessária um lista
    ordenada dos mesmos esta tem de ser percorrida na sua totalidade. Esta
    decisão centrou-se no facto de nenhuma única ordenação se apresentar
    particularmente vantajosa, face às demais.

\end{document}
