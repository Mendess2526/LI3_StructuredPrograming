\documentclass[11pt,a4paper]{report}
\usepackage[utf8]{inputenc}
\usepackage[portuguese]{babel}
\usepackage{titlesec}
\usepackage{graphicx}
\usepackage{float}
\usepackage{array}
\usepackage{multirow}
\usepackage{geometry}
\usepackage{minted}
\usepackage{pdflscape}
\usepackage[titletoc]{appendix}
\geometry{
 a4paper,
 top=3cm,
 bottom=2.5cm,
 left=3cm,
 right=2.5cm
}
\titleformat{\chapter}{\normalfont\huge}{\thechapter.}{10pt}{\huge}
\begin{document}
\begin{titlepage}
    \center
    \vspace*{4.0cm}
    {\huge {\bf Universidade do Minho}}\\[0.4cm]
    \vspace{3.0cm}
    \textsc{\huge{Processamento de xml}}\\[0.5cm] %TODO melhorar isto
    \vspace{3.0cm}
    \textsc{\huge{Mestrado Integrado em Engenharia Informática}}\\[0.5cm]
    \vspace{2.0cm}
    \textsc{Laboratorios de Informática 3}\\[0.5cm]
    \textsc{(2º Ano, 2º Semestre, 2017/2018)}\\[0.5cm]
    \vspace{1.5cm}
    \begin{flushleft}
        A79003 \,\,\,Pedro Mendes Félix da Costa
        \vspace{0.2cm}

        AXXXXX \,\,\,Barbara Andreia Cardoso %TODO acabar nome
        \vspace{0.2cm}

        AXXXXX \,\,\,Ana Catarina %TODO
    \end{flushleft}
        \vspace{1cm}
    \begin{flushright}
        Braga

        Abril 2018
    \end{flushright}

\end{titlepage}

\chapter{Resumo}


\tableofcontents %indice

\chapter{Introdução}


\chapter{Descrição do Problema}


    \section{Users}


    \section{Posts}


        \subsection{Questions}


        \subsection{Answers}


\chapter{Solução do Problema}


    \section{Estruturas de Dados}


\chapter{Conclusões e Trabalho Futuro}


\end{document}
