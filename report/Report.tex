\documentclass[10pt,a4paper]{report}
\usepackage[utf8]{inputenc}
\usepackage[portuguese]{babel}
\usepackage{titlesec}
\usepackage{graphicx}
\usepackage{indentfirst}
\usepackage{float}
\usepackage{array}
\usepackage{multirow}
\usepackage{geometry}
\usepackage[cache=false]{minted}
\usepackage{pdflscape}
\usepackage[titletoc]{appendix}
\geometry{
 a4paper,
 top=1cm,
 bottom=1cm,
 left=3cm,
 right=3cm
}
\addto\captionsportuguese{% Replace "english" with the language you use
      \renewcommand{\contentsname}%
          {Índice}%
}
\titleformat{\chapter}{\normalfont\huge}{\thechapter.}{10pt}{\huge}
\begin{document}
\begin{titlepage}
    \center
    \vspace*{4.0cm}
    {\huge {\bf Universidade do Minho}}\\[0.4cm]
    \vspace{3.0cm}
    \textsc{\huge{Processamento de xml}}\\[0.5cm] %TODO melhorar isto
    \vspace{3.0cm}
    \textsc{\huge{Mestrado Integrado em Engenharia Informática}}\\[0.5cm]
    \vspace{2.0cm}
    \textsc{Laboratorios de Informática 3}\\[0.5cm]
    \textsc{(2º Ano, 2º Semestre, 2017/2018)}\\[0.5cm]
    \vspace{1.5cm}
    \begin{flushleft}
        A79003 \,\,\,Pedro Mendes Félix da Costa
        \vspace{0.2cm}

        A80453 \,\,\,Bárbara Andreia Cardoso Ferreira
    \end{flushleft}
        \vspace{1cm}
    \begin{flushright}
        Braga

        Abril 2018
    \end{flushright}

\end{titlepage}

\tableofcontents

\chapter{Introdução}
    Este trabalho tem como objetivo 

\chapter{Descrição do Problema}
    Analisando as queries decidimos que seria necessário representar
    as seguintes entidades:

    \section{Utilizadores}
    Para represantar os \textbf{utilizadores} guardamos os seguintes atributos:
    \begin{itemize}
            \item Id
            \item Biografia
            \item Nome
            \item Reputação
            \item Número de posts
            \item Lista dos posts
    \end{itemize}

    \section{Posts}
    Para represantar os \textbf{posts} dividimos os atributos por duas
     sub estruturas, cujos atributos são:
        \subsection{Questões}
        \begin{itemize}
            \item Id
            \item Score
            \item Data
            \item Título da questão
            \item Id do autor da questão
            \item Nome do autor da questão
            \item Número de respostas
            \item Lista das respostas
            \item Tags
        \end{itemize}

        \subsection{Respostas}
        \begin{itemize}
            \item Id
            \item Score
            \item Data
            \item Número de comentários
            \item Id do autor da resposta
            \item Id da questão a que responde
            \item Nome do autor da resposta
            \item Referência da questão a que responde
        \end{itemize}


\chapter{Tipo Concreto de Dados}
    Para armazenar as entidades descritas acima foi implementado um TCD que
    as armazena de diferentes formas.
    \begin{minted}{C}
struct TCD_community{
    QUESTIONS_HTABLE questions;
    ANSWERS_HTABLE answers;
    SO_USERS_HTABLE users;
    TAGS_HTABLE tags;
    CALENDARIO calendarioQuestions;
    CALENDARIO calendarioAnswers;
};
    \end{minted}
    \section{Hashtables}
        Todas as entidades são armazenadas numa hashtable pois para todas são
        necessárias pesquisas por id (ou nome no caso das tags).
        \subsection{Tags}
        A hashtable das tags serve para criar uma associação "Nome Id" visto que
        as questões guardam uma lista com os nomes das tags e para responder à
        query 11 é necessário obter os ids das mesmas.

    \section{Calendário}
        Foi pensada a utilização de árvores binárias ordenadas por data,
        mas isto foi considerado ineficiente por várias razões. Foi então,
        concebida uma estrutura a qual demos o nome de \textbf{Calendário}.

        Esta estrutura permite:
        \begin{itemize}
                \item Guardar qualquer objeto desde que seja
                      passada uma data associada ao mesmo.
                \item Iterar sobre os objetos dado um intervalo de tempo
                      por ordem cronológica normal ou inversa, conforme a
                      ordem dos argumentos.
        \end{itemize}
        \subsection{DateTime}
        Para a implementação deste foi necessário extender o \textbf{Date} para
        incluir a hora, minuto, segundo e milissegundo do post. Foi então que
        foi criado o \textbf{DateTime}
    \begin{minted}{C}
struct _dateTime{
    int year;
    int month;
    int day;
    int hours;
    int minutes;
    int seconds;
    int milisseconds;
};
    \end{minted}
        


\chapter{Modularização Funcional e Resolução das queries}
    Para aceder aos dados da estrutura principal foi definida uma API
    simples que permite:
    \begin{itemize}
        \item Pesquisas por id de questões, respostas, utilizadores.
        \item Pesquisas por tag do id desta.
        \item Pesquisa de listas ordenadas por qualquer critério de utilizadores,
                questões e respostas.
        \item Pesquisa de questões filtradas por qualquer critério.
    \end{itemize}

    Com estas funções a resolução da maioria das queries mostrou-se
    trivial.

    Para a resolução da query 11 foi implementada uma estrutura para contar tags,
    para que esta contagem destas fosse eficiente.
\chapter{Conclusões e Trabalho Futuro}


\end{document}